\section{Special cases}\label{sec:explanation}

\subsection{How to create new repository}

The diagram~\ref{fig:ex-create-repo} explains how administrator creates new repository with replication
on predefined storage nodes by specifying its locator IDs:
\begin{description}
  \item[1] Administrator sends request to the dashboard to create new repository \code{R} at nodes with locators
    \code{L}
  \item[2] Dashboard service performs a lookup of \code{R} in metadata storage
  \item[3] Metadata storage returns locator IDs \code{L'} of \code{R} repository if any
  \item[4, 5] If \code{L'} locators are not emmpty, dashboard returns error to administrator
  \item[6] If no node locator is associated with repository \code{R}, then dashboard asynchronously sends
    requests to each node with locator ID \code{l} from list \code{L}
  \item[7] Dashboard asks node \code{l} to start manage repository \code{R}, and specify other node locator IDs;
    The node \code{l} try to organize consensus between nodes \code{L}, and leader asks all other members
    to manage repository \code{R}
  \item[8] On success, node \code{l} updates metadata with mapping of repository \code{R} hash to
    node locator \code{l}
  \item[9] Meanwhile, the dashboard is performing query requests to the metadata storage to get locator IDs of
    repository \code{R}; Metadata storage returns IDs \code{L'}.
  \item[10] In case if \code{L'} is a subset of \code{L} and the size of \code{L'} is greater or equal to
    half size of \code{L} plus 1 (\code{size(L') >= size(L)/2 + 1})
  \item[11] Dashboard finishes with success status
  \item[12] If waiting timeout is reached
  \item[13] Dashboard finishes with error status
  \item[14] Dashboard returns with success or error status to administrator

\end{description}
\begin{figure}
  \begin{center}
    \begin{tikzpicture}
      \begin{umlseqdiag}
        \umlactor[no ddots]{Administrator}
        \umlcontrol[no ddots]{Dashboard}
        \umlmulti[no ddots]{Nodes}
        \umldatabase[no ddots]{Metadata}

        \begin{umlcall}[op={1: create(R, L)}]{Administrator}{Dashboard}
          \begin{umlcall}[op={2: lookup(R)}]{Dashboard}{Metadata}
            \begin{umlcall}[op={3: L'}, type=return]{Metadata}{Dashboard}
            \end{umlcall}
          \end{umlcall}

          \begin{umlfragment}[type=ALT, name=err-already-exist, inner xsep=3,
            label={4: !empty(L')}]
            \begin{umlcall}[op={5: error}, type=return]{Dashboard}{Administrator}
            \end{umlcall}
            \umlfpart[else]
            \begin{umlfragment}[type=LOOP, inner xsep=7, label={6: for l in L}]
              \begin{umlcall}[op={7: l.manage(R, L)}, type=asynchron]{Dashboard}{Nodes}
                \begin{umlcall}[op={8: update(R, l)}, type=asynchron]{Nodes}{Metadata}
                \end{umlcall}
              \end{umlcall}
            \end{umlfragment}
            \begin{umlfragment}[type=LOOP]
              \begin{umlcall}[op={9: lookup(R)}, return={L'}]{Dashboard}{Metadata}
              \end{umlcall}
              \begin{umlfragment}[type=OPT, label={10: contains(consensus(L), L')}]
                \begin{umlcallself}[op={11: break(success=true)}]{Dashboard}
                \end{umlcallself}
              \end{umlfragment}
              \begin{umlfragment}[type=OPT, label={12: timeout}]
                \begin{umlcallself}[op={13: break(success=false, err=timeout)}]{Dashboard}
                \end{umlcallself}
              \end{umlfragment}
            \end{umlfragment}
            \begin{umlcall}[type=return, op={14: (success, err)}]{Dashboard}{Administrator}
            \end{umlcall}
          \end{umlfragment}
        \end{umlcall}
      \end{umlseqdiag}
    \end{tikzpicture}
  \end{center}
  \caption{Create repository workflow diagram}
  \label{fig:ex-create-repo}
\end{figure}


\subsection{How the node finds the replica with git repository}

Each node (both front-end and back-end nodes) has metadata exchange and discovery protocol components,
both components rely on network module and locators systems.

Stakeholders:
\begin{description}
  \item[Node] A node which performs repository lookup operation
  \item[RDB] Regional database (caches repository hash table in local region specific cache table)
  \item[DHT] Distributed hash table nodes, can perform global query lookup for node address by repository hash
\end{description}

The legend: \code{r} --- repository hash to lookup; \code{(l, a)} --- repository coordinates,
pair of replica locator IDs and network address, see figure~\ref{fig:ex-node-lookup} for details.

Workflow:
\begin{enumerate}
    \item Node checks local cache populated by local peer discovery broadcasts
      and updated anually after success query operation
    \item If coordinates were not found in cache, node goes to the next lookup layer
    \item Node queries regional database to find repository coordinates
    \item If coordinates were not found in database, goes to the next lookup layer
    \item Node queries DHT nodes using lookup algorithms, e.g. Kademlia.
    \item On success, node updates regional database with actual information
    \item Node stores actual information in local cache
\end{enumerate}

\begin{figure}
  \begin{center}
    \begin{tikzpicture}
      \begin{umlseqdiag}
        \umlobject[no ddots]{Node}
        \umldatabase[no ddots]{RDB}
        \umlmulti[no ddots]{DHT}
        \begin{umlcallself}[op={1: getCache(r)}]{Node}
        \end{umlcallself}
        \begin{umlfragment}[type=OPT, label={2}]
          \begin{umlcall}[op={3: query(r)}, return={(l, a)}]{Node}{RDB}
          \end{umlcall}
          \begin{umlfragment}[type=OPT, label={4}]
            \begin{umlcall}[op={5: query(r)}, return={(l, a)}, dt=7]{Node}{DHT}
            \end{umlcall}
            \begin{umlcall}[op={6: update(r, (l, a))}]{Node}{RDB}
            \end{umlcall}
          \end{umlfragment}
          \begin{umlcallself}[op={7: putCache(r, (l, a))}]{Node}
          \end{umlcallself}
        \end{umlfragment}
      \end{umlseqdiag}
    \end{tikzpicture}
  \end{center}
  \caption{Node discovery and metadata exchange components}
  \label{fig:ex-node-lookup}
\end{figure}

\subsection{How GitLab pushes to DeGitX via Gitaly front-end}

GitLab communicates with DeGitX storage via Gitaly front-end. The front-end
exposes Gitaly gRPC~API.\@ GitLab performs push with two RPC calls:
\begin{description}
  \item[InfoRefsReceivePack] --- to fetch latest repository git references before pushing
  \item[PostReceivePack] --- for add new git objects to the repository
\end{description}

At figure~\ref{fig:gitlab-push-gitaly}, front-end updates repository lookup routing table
to request git references from one of the replica and send new git objects to back-end nodes.
See~\ref{sec:appendix-a} to become familiar with git internals.

This diagram explains only communication between GitLab and DeGitX Gitaly front-end,
to understand how the front-end can read and write git data from/to back-ends, see
other diagrams below.

\begin{figure}
\centering
  \begin{tikzpicture}
    \begin{umlseqdiag}
      \umlcontrol[no ddots]{git}
      \umlobject[no ddots]{GitLab}
      \umlmulti[no ddots]{DeGitX-FE}
      \umldatabase[no ddots]{Meta}
      \umlmulti[no ddots]{DeGit-BE}
      \begin{umlcall}[op=git-push, return=success]{git}{GitLab}
        \begin{umlcall}[op={InfoRefsReceivePack(R)}]{GitLab}{DeGitX-FE}
          \begin{umlcall}[op={lookup(R)}, return={L}]{DeGitX-FE}{Meta}
          \end{umlcall}
          \begin{umlfragment}[type=ALT, inner xsep=6, label={\code{empty(L)}}]
            \begin{umlcall}[type=return, op={err-not-found}]{DeGitX-FE}{GitLab}
            \end{umlcall}
          \umlfpart[else]
            \begin{umlcallself}[op={\code{update\_cache(R, L)}}]{DeGitX-FE}
            \end{umlcallself}
            \begin{umlcall}[op={L$_{lb}$ \code{read\_refs(R)}}, return={\code{refs}}]{DeGitX-FE}{DeGit-BE}
            \end{umlcall}
            \begin{umlcall}[type=return, op={\code{refs}}]{DeGitX-FE}{GitLab}
            \end{umlcall}
          \end{umlfragment}
        \end{umlcall}
        \begin{umlfragment}[type=OPT, inner xsep=5, label={found}]
          \begin{umlcall}[op={\code{PostReceivePack(R, bin)}},return={success}]{GitLab}{DeGitX-FE}
            \begin{umlcall}[op={L$_{lb}$ \code{receive\_pack(R, bin)}}]{DeGitX-FE}{DeGit-BE}
            \end{umlcall}
          \end{umlcall}
        \end{umlfragment}
      \end{umlcall}
    \end{umlseqdiag}
  \end{tikzpicture}
\caption{%
  User pushes git data to repository \code{R} via Gitlab using \code{git-push} command;%
  Gitlab calls \code{InfoRefsReceivePack} of DeGitX front-end to get latest information%
  about repository \code{R}; front-end perorms lookup in metadata storage,%
  if repository was not found, front-end returns error for gRPC~call;%
  otherwise, it updates local cache with repository locators \code{L}, and reads git references from%
  any load-balanced replica \code{L$_{lb}$}; the front-end returns git data to GitLan server;%
  on success, GitLab sends \code{PostReceivePack} gRPC~call to DeGitX front-end with data to push \code{bin};%
  front-end write data via \code{receive\_pack} git method to any replica \code{L$_{lb}$};%
  (see other examples to understand how back-end nodes updates repository data on push).%
}\label{fig:gitlab-push-gitaly}
\end{figure}

\subsection{How the front-end pushes data to the storage}

The diagram~\ref{fig:ex-push-to-repo} explains how the front-end can push data to git repository.
Front-end ``FE'' chooses one repository ``R'' replica to communicate with (``BE$_{1}$'' in example diagram),
this node is responsible to talk to the leader or initiate leader election if no leader exist.
The leader is responsible to replicate push command and data across consensus of replicas
and notify BE$_{1}$ node on success.

On this diagram, the stakeholders are:
\begin{description}
  \item[FE] --- front-end of DeGitX (load-balancer, multiplexer), who received the request to push new data
  \item[BE$_{1}$] --- the back-end of DeGitX (git storage), chosen by front-end load-balancer to interact with
  \item[BE$_{o}$] --- other back-end replicas, containing target repository to push
  \item[Meta] --- Metadata storage (LPD, DB, DHT)
\end{description}

The workflow:
\begin{description}
  \item[1] Client calls RPC of FE to push commits \code{c} to repository \code{r}
  \item[2] FE performs lookup in metadata to find lcoators \code{L[]} of repository \code{r}
  \item[3] FE calls RPC of any storage node from \code{L[]} list to push commits \code{c} to
    repository \code{r}
  \item[4] BE RPC receiver organizes leader-election across replicas of repository \code{r}
    and asks a leader to add new commit \code{c} to repository git storage
  \item[5] At least the half plus one replica performs update and commit new changes
\end{description}

\begin{figure}
  \centering
  \begin{tikzpicture}
    \begin{umlseqdiag}
      \umlobject[no ddots]{FE}
      \umlmulti[no ddots]{BE$_{1}$}
      \umlmulti[no ddots]{BE$_{n}$}
      \umldatabase[no ddots]{Meta}
      \begin{umlcall}[op={lookup(R)}, return={L}]{FE}{Meta}
      \end{umlcall}
      \begin{umlcallself}[op={BE$_{1}$ = any(L)}]{FE}
      \end{umlcallself}
      \begin{umlcall}[op={receive\_pack(R, d)}]{FE}{BE$_{1}$}
        \begin{umlcallself}[op={l = leader(R)}]{BE$_{1}$}
        \end{umlcallself}
        \begin{umlfragment}[type=ALT, name=repl, inner xsep=6, label={l in BE$_{n}$}]
          \begin{umlcall}[op={receive\_pack(R, d)}, return={confirmed}]{BE$_{1}$}{BE$_{n}$}
            \begin{umlcall}[op={append\_entries}]{BE$_{n}$}{BE$_{1}$}
            \end{umlcall}
          \end{umlcall}
        \umlfpart[else]
          \begin{umlcallself}[op={elect}]{BE$_{1}$}
          \end{umlcallself}
          \begin{umlcall}[op={append\_entries}, return={confirmed}]{BE$_{1}$}{BE$_{n}$}
          \end{umlcall}
        \end{umlfragment}
        \umlnote[x=11.5, y=-6.5]{repl}{%
          Add entries to existing leader, or propose itself as leader candidate%
        }
      \end{umlcall}
    \end{umlseqdiag}
  \end{tikzpicture}
  \caption{%
    Front end FE performs query lookup in metadata for repository replicas,%
    chooses any available (BE$_{1}$), and sends \code{receive\_pack} request with pushed data%
    to replica. Replica checks current term leader and asks a leader to add pack to replicated log,%
    in case if leader doesn't exist, the replica initiate new leader election and proposes itself%
    as leader candidate. On success, the replica adds new pack to replicated log via \code{append\_entries}%
    call, and commits the data.%
  }\label{fig:ex-push-to-repo}
\end{figure}

\subsection{How the client fetches git data from repository}

\subsection{How new replica node connected to the cluster}

\todo{explain all questions}
