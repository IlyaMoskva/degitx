\documentclass[acmlarge, screen, nonacm]{acmart}
\usepackage[utf8]{inputenc}
\usepackage{float}
\usepackage{setspace}
\usepackage{pgfplots}
\usepackage{graphicx}
\usepackage{fancyvrb}
\usepackage{listings}
\usepackage{xcolor}
\usepackage{hyperref}
  \hypersetup{colorlinks=true,allcolors=blue!40!black}
\setlength{\topskip}{6pt}
\setlength{\parindent}{0pt} % indent first line
\setlength{\parskip}{6pt} % before par

\title{GitLab commit protocol implementation}

\author{Ivan Bukhtiyarov}
\email{TODO: put your email address here}

\author{Kirill Chernyavskiy}
\email{g4s8.public@gmail.com}

\acmBooktitle{none}
\acmConference{none}
\editor{none}

\begin{document}

\begin{abstract}
  GitLab claims it uses three-phase commit (3PC) protocol for managing atomic transactions.
  (TODO: add references to GitLab tickets). But 3PC has many different implementations and
  additional extensions (such as recovery and termination protocols): it means that 3PC could
  be implemented differently depends on business requirements. In this report we analyze what
  components of 3PC is used by GitLab and how it achieves strong consistency, by describing
  the whole workflow of each git push command from client to replicas and investigating
  extension protocols for dealing with node and network failures.
\end{abstract}

\maketitle

\section{Introduction}

TODO: add introduction, describe how GitLab achieves strong consistency (summary).

\section{Workflow}

TODO: the workflow of GitLab push: from client to Gitaly nodes.

\section{3PC}

TODO: what version of 3PC is used for GitLab. How GitLab achieves atomic commit protocol properties:
stability, consistency, non-triviality, non-blocking described in
``Consensus on transaction commit by J.Gray and L.Lamport''.

\section{Recovery protocol}

TODO: how the GitLab handles node failures both RMs and TMs.

\section{Termination protocol}

TODO: what is the termination strategy to finish the transaction.


TODO: add references (see bibtex).
\end{document}
